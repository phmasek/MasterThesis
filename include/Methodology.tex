\iffalse \bibliography{include/backmatter/magnus,include/backmatter/philip} \fi
\chapter{Methodology}

In order to answer the research questions, an experiment was carried out following the guidelines presented by Shull et al. 

Two sets of runs for each RQ where 1 is without load and 1 is with load. 
	- we want to find wether there is a correlation between enviroment and performance depending on load 
Why not memory? Because is a RT system you allocate static memory to each component before execution. 

\section{Experiment}
The aim 

\subsection{Goals}
The goal of the experiment is to systematically evaluate the scheduling precision of running three sample applications in different execution enviroments. The execution enviroments are. 

Running application A in Linux using VanillaKernel version 3.18.25
Running application A within a Docker container using Vanilla Kernel version 3.18.25
Running application A in Linux using RT kernel 
Running application A within a Docker container using RT Kernel 

\subsection{Preparing the Target System}
Before the test can be carried out, the target system needs to be perpared with different systems. 

\begin{table}[H]
\centering
\label{my-label}
\begin{tabular}{|l|l|}
\hline
\textbf{Component} & \textbf{Specification}                                                                                                                                                                                                                                                                          \\ \hline
Processor          & Intel® Core™ i7 3517UE 1.7 GHz                                                                                                                                                                                                                                                                  \\ \hline
Memory             & 4GB DDR3 1333/1600 SODIMM                                                                                                                                                                                                                                                                       \\ \hline
Video              &                                                                                                                                                                                                                                                                                                 \\ \hline
Storage Device     & 2.5" SATA HDD x 1                                                                                                                                                                                                                                                                               \\ \hline
Serial Interfaces  & \begin{tabular}[c]{@{}l@{}}USB type A x 2 for USB 2.0\\ USB type A x 2 for USB 3.0\\ DB-9 x 2 for RS-232/422/485 x 2\\ DB-9 x 4 for RS-232 x 4\\ Isolated Digital Input/Output x 10 pins (DI x 4, DO x 4)(3KV) \\ Isolated DB-9 x 2 for RS-232/422/485 x 2 (3KV, jumper selection)\end{tabular} \\ \hline
\end{tabular}
\caption{My caption}
\end{table}


\subsubsection{Hardware}
The tests presented in this report will be executed on a




\subsubsection{Building GNU/Linux}
\subsubsection{Configuring the Kernel}

\subsection{Experimental Units}
RQ1 - Pi Component, RQ2 Pi/IO Component (measurement points)

\subsection{Experimental Material}

\subsection{Hypotheses, Parameters and Variables}


\subsection{Design}
Quasi design


\subsection{Procedure}
RPI data collector 

\subsection{Analysis Procedure}
