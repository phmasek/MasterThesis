\iffalse \bibliography{include/backmatter/magnus,include/backmatter/philip} \fi
\chapter{Introduction}
%funneling from the world to the actual problem 
Cyber-phyiscal systems (CPS) are increasing becoming more depent upon in todays time. Soon enough cyber physical systems will enable us to have more time for ourselves, where mundane tasks are slowly removed such as automated parking or highway driving. Such systems are very complex, involving multiple software and hardware components, of differenty types and architectures. 

(Pure software products) Software deployment for typical software is straightforward and robust. Continuous Intergration and Continuous Deployments are software engineering concepts that have recently become more important. The societal transformation from a product to service economy has given rise to a number of tools and methodologies to bring more value to customers in a shorter ammount of time. To do this you must deploy features fast and easily. Products are required to move from business requirements to delivery as soon as possible. For cyber physical systems it is more challenging. 

Software deployment for cyber physical systems is difficult when there are many things to put into consideration. You typically have contrained resources since you cannot scale up hardware substantially on a vehicle, its physically not possible. So a deployment strategy that is lightweight for the reciever is required. 


Virtualisation is an important concept to continous deployment, since ... 
Real-time, overhead, process isolation, etc. 


This study aims to understand if lightweight virtualisation can be used on self driving vehicles when real-time requirements are needed. 

CPS it similar to IOT sharing the same architecture, between physical and computational elements

% CI and CD are important software engineering concepts to enable continuous delivery of features to customers. Well implemented and maintained CI/CD is crucial to allow a company to adapt to changing market demands and evolving customer requirements. CD of features is also becoming increasing important in mechatronical domains, where a complex software system has to safely and reliably inter-operate hardware and sensors. 

%The supportive software engineering processes therefor are called continuous integration, continuous deployment, and continuous experimentation.While these process are growling well established in pure software products,aiming for similar opportunities in embedded and mechatronics systems is more complex due to real-time and safety constraints to just name a few

%---------------------------------------------------%
\section{Background}
This section introduces the technologies and concepts related to this paper.

%latency mitigated by modifying operating sytem to provide more determinism 
%linux foundation announced fully support rt linux 
%virtualization can be done with fully-fledged Virtual Machines or lightweight containers. VMs  incure higher ovhead but better protection and so are not suitable for autonomous vehicles.  Therefore we adopt containers and strive to minimize latency by real-time Linux kernel
%containers very efficiently share and utilize CPU and memory resources, 


\subsection{Cyber Physical Systems}
\subsection{Software Deployment}
\subsection{Container-Based Virtualization}
%process isolation and application portability
\subsection{Task Scheduling} %kernel and cpu scheduler
compact middleware OpenDaVINCI written in standard C++, can be used on a variety of POSIX OS. 
we use ODV to have a lean, portable and high-performance hardware and OS abstraction layer for typical programming idioms like concurrency, data storage and communication
\subsection{Real Time System and Scheduling Precision}
%---------------------------------------------------%



%---------------------------------------------------%
\section{Problem Domain \& Motivation}
% This is a dedicated section for problem and motivation, you should not deviate too much from our domain and context
%---------------------------------------------------%



%---------------------------------------------------%
\section{Research Goal \& Research Questions}
% We need rationales to explain what we expect from these RQs. Further, it would be nice to have some preparation sentences that would justify asking these questions and not others
%---------------------------------------------------%



%---------------------------------------------------%
\section{Contributions}
% The contribution made to the research community by answering the RQs.} 
% Many studies have analyzed virtual machines and containers to compare their performance cite. 

\section{Scope}
% In this section we will introduce the scope which is self-driving vehicles. The hardware we are using is in-line with actual hardware used in autonomous vehicles. The software development architecture used for our experimental units have been adopted in research projects involving the actual development of autonomous vehicles. We do not aim for our results to be valid in other types of autonomous systems, such as drones.

\section{Structure of the thesis}
% Summary: In this section we will introduce a typical outline of the paper.}
%---------------------------------------------------%