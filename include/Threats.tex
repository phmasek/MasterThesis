\iffalse \bibliography{include/backmatter/magnus,include/backmatter/philip} \fi
\chapter{Threats to Validity}


\textit{Summary: In this section we will introduce potential weakenesses we are aware of by following the guidelines of Runeson \& Höst, Feldt \& Magazinius.}

% How did we validate the results? 
	% statistical analysis is powerful 
	% according to the one paper "an updated performance.." running with no load results in no difference between. Our results are inline with those findings 
	% there exists papers that state using docker for IO should be taken with caution. How did our results perform? 
	% every run of tests is automated, perparing the target system is automated so we can ensure that results between eachother are the same

% How was the data actually validated?
	% we store data over serial line that uses a dedicated interrupt line so that storing the results does not have an effect on the experimental units
	% for the RT kernel, we ran the same test with our kernel and with Capocasa to see if there is a difference. Also, we use OSADL configuration file which is a organisation that benchmarks Linux RT kernel 



% Some papers say that operating systems are not good for measuring time, however, there are many other papers that use the same method as us but in a more simplistic way (without using any form of middleware (ODV))

% kernel configuration (did we do it right? - we did compare our results to Capocasa (a well usd RT Kernel) and there was no significant difference)

% Data comming from the RPI is valid since we know how many measurement points we expect 
% 