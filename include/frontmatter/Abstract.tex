% CREATED BY DAVID FRISK, 2015
Exploring Light-Weight Virtual Containers for Real-Time Systems in the Context of Self-Driving Vehicles\\
Philip Masek\\
Magnus Thulin\\
Department of Computer Science and Engineering\\
Chalmers University of Technology \setlength{\parskip}{0.5cm}

\thispagestyle{plain}			% Supress header 
\setlength{\parskip}{0pt plus 1.0pt}
\section*{Abstract}
Delivering new software features in a continuous fashion has become a competitive advantage for organisations operating in the web domain. Being able to deliver new features to customers on a regular basis allows organisations to rapidly respond to change in customer requirements and to verify customer value. Software development in the domain of web applications differ greatly in comparison to embedded, cyber-physical systems which are tightly coupled to hardware, electronics and mechanics. A cyber-physical system (CPS) can benefit from a platform that enables the continuous deliver of new features. Virtual machines is a popular method for software deployment where applications are sand-boxed and pre-installed in a highly portable environment. This study contributes to the research community by understanding the performance overhead of using virtual containers as a deployment platform for CPSs which are highly sensitive to timing delays. Methods of experimentation are used to understand the timing behaviour of two sample applications realised with the development architecture for CPSs, OpenDaVINCI. Sample applications are run in various deployment and execution environments where a real-time enabled Linux kernel is used. Hypotheses testing and statistical analysis is performed on timestamps extracted from the sample applications, where results show that the virtual container manager Docker achieves near native performance when executing applications in a virtual environment in comparison to native execution.  The experiment is executed in a controlled environment where the results are validated by adapting the experiment on a self-driving vehicle that participated in the Grand Cooperative Driving Challenge 2016 held in the Netherlands. This research concludes that Docker together with a real-time enabled kernel is a deployment platform good candidate for vehicular CPSs. 


%1. current state of art identigying a particular problem
%2. the contribution to improving the situation
%3. the specific results and main idea behind it
%4. how the result is demonstrated or defended



% KEYWORDS (MAXIMUM 10 WORDS)
\vfill
Keywords: Deployment, self-driving vehicles, cyber-physical systems, virtual containers, Docker, real-time systems.

\newpage				% Create empty back of side
\thispagestyle{empty}
\mbox{}