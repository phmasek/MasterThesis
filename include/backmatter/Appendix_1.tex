\chapter{Source Code}
\section{Pi Component}


\begin{lstlisting}[frame=single,label={code:pi}]
while (getModuleStateAndWaitForRemainingTimeInTimeslice() == odcore::data::dmcp::ModuleStateMessage::RUNNING) {
	// Second measurement point
    odcore::data::TimeStamp::writeNanoToSerial("2");
    odcore::data::TimeStamp start, end;

    // Pi algorithm variable are
    // reset after each timeslice.
    long double tempPi;
    long double pi     = 4.0;
    int i              = 1;
    int j              = 3;
    float oDuration    = 0;

    while (true) {

        // Calculate pi
        tempPi = static_cast<double>(4)/j;
        if (i%2 != 0) {
            pi -= tempPi;
        } else if (i%2 == 0) {
            pi += tempPi;
        }
        i++;
        j+=2;


        // Occupy for a certain duration
        end = odcore::data::TimeStamp();
        oDuration = end.toNanoseconds()-start.toNanoseconds();
        if (oDuration >= occupy)
            break;
    }
    // Third measurement point
    odcore::data::TimeStamp::writeNanoToSerial("3");
    if (piTimes==duration)
        return odcore::data::dmcp::ModuleExitCodeMessage::OKAY;
}
\end{lstlisting}

\newpage
\section{Pi/IO Component}
\begin{lstlisting}[frame={single},label={code:pi-io}]
while (getModuleStateAndWaitForRemainingTimeInTimeslice() == odcore::data::dmcp::ModuleStateMessage::RUNNING) {
	// Second measurement point
    odcore::data::TimeStamp::writeNanoToSerial("2");
    odcore::data::TimeStamp start, end;

    if (m_camera.get() != NULL) {
        odcore::data::image::SharedImage si = m_camera->capture();
        // Third measurement point
        odcore::data::TimeStamp::writeNanoToSerial("3");
        odcore::data::Container c(si);
        distribute(c);
        // Fourth measurement point
        odcore::data::TimeStamp::writeNanoToSerial("4");
    }



    // Pi algorithm variable are
    // reset after each timeslice.
    long double tempPi;
    long double pi     = 4.0;
    int i              = 1;
    int j              = 3;
    float oDuration    = 0;

    while (true) {

        // Calculate pi
        tempPi = static_cast<double>(4)/j;
        if (i%2 != 0) {
            pi -= tempPi;
        } else if (i%2 == 0) {
            pi += tempPi;
        }
        i++;
        j+=2;


        // Occupy for a certain duration
        end = odcore::data::TimeStamp();
        oDuration = end.toNanoseconds()-start.toNanoseconds();
        if (oDuration >= occupy)
            break;
    }

    // Fifth measurement point
    odcore::data::TimeStamp::writeNanoToSerial("5");
    if (piTimes==duration)
        return odcore::data::dmcp::ModuleExitCodeMessage::OKAY;
}
\end{lstlisting}


\begin{lstlisting}[frame=single,label={code:serialtime}]
std::shared_ptr<odcore::wrapper::SerialPort> TimeStamp::m_serialPort = NULL;
void TimeStamp::setupSerial(const string port, uint32_t baud_rate) {
    std::shared_ptr<odcore::wrapper::SerialPort> serial(odcore::wrapper::SerialPortFactory::createSerialPort(port, baud_rate));
    TimeStamp::m_serialPort = serial;
}

const string TimeStamp::writeNanoToSerial(const char* measurementID) {
    TimeStamp ts = TimeStamp();
    try {
        stringstream s;
        s << measurementID << ";" << ts.toNanoseconds() << "\r\n";
        if (m_serialPort) {
            m_serialPort->send(s.str());
        } else {
            return "Serial port not defined";
        }
        return "Successfully wrote toNanosecond() to serial";
    }
    catch(string &exception) {
        return exception;
    }
}

const string TimeStamp::writeMicroToSerial(const char* measurementID) {
    TimeStamp ts = TimeStamp();
    try {
        stringstream s;
        s << measurementID << ";" << ts.toMicroseconds() << "\r\n";
        if (m_serialPort) {
            m_serialPort->send(s.str());
        } else {
            return "Serial port not defined";
        }
        return "Successfully wrote toMicrosecond() to serial";
    }
    catch(string &exception) {
        return exception;
    }
}

const string TimeStamp::writeStringToSerial(const char* measurementID) {
    TimeStamp ts = TimeStamp();
    try {
        stringstream s;
        s << measurementID << ";" << ts.toString() << "\r\n";
        if (m_serialPort) {
            m_serialPort->send(s.str());
        } else {
            return "Serial port not defined";
        }
        return "Successfully wrote toString() to serial";
    }
    catch(string &exception) {
        return exception;
    }
}

const string TimeStamp::writeMessageToSerial(const string message) {
    try {
        if (m_serialPort) {
            m_serialPort->send(message);
        } else {
            return "Serial port not defined";
        }
        return "Successfully wrote message to serial";
    }
    catch(string &exception) {
        return exception;
    }
}
\end{lstlisting}
