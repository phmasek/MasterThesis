\iffalse \bibliography{include/backmatter/magnus,include/backmatter/philip} \fi
\chapter{Conclusion \& Future Work} \label{section:conclusion}

This research sought to understand the impact Docker brings when utilised in the deployment strategy for systems which enable self-driving capabilities in vehicles. The research was conducted in collaboration with research project held in collaboration between Chalmers and SAFER who are exploring the development of a self-driving truck. The results from this research acts as support in the decision making process of using Docker containers for software deployment for further software development on the self-driving truck project. Previous literature \cite{p6,p10,p3,p4,p7,p9} show promising results for containers as a deployment platform as their presented results show negligible overhead introduced by utilising Docker in the execution environment of the applications explored. However, previous research has not been able to build a case for the decision making specific for the domain of self-driving vehicles where the domain stresses the real-time requirements of vehicular cyber-physical systems. While the research conducted in \cite{p1} have explored how Docker behaves using cyclic test in a cloud computing environment with real-time requirements, it has not explored how Docker behaves in the context of a complete cyber-physical system. This is the gap this research has explored.\\

The findings from this research are in-line with previous literature and show that Docker introduces negligible overhead to the scheduling precision, input and output performance to the cyber-physical system in both the controlled environment as well as the environment of the self-driving truck at Chalmers Revere lab. Furthermore, the results convey that choosing a correct \textit{kernel} for the system carry heavier importance in comparison to choosing Docker or native execution for the deployment strategy. As the results show that the kernel type has a larger impact on the scheduling precision. The results show a considerably more deterministic environment when utilising an RT\_preempt kernel in comparison to a vanilla kernel. This argues that Docker is not the predominant factor impacting the scheduling or IO performance for when designing the correct execution environment for a self-driving vehicle.\\

This research has focused on the specific context of implementing Docker as a deployment strategy for self-driving vehicles. The design of this research has aimed to find an understanding of how Docker impacts scheduling precision and camera and disk performance for the contexts used. Analysis has been carried out to understand how these performance factors behave while switching between different treatments, such as system load and kernel version. The performance factors scheduling precision and camera input and disk output performance build a foundation of understanding the impact a Docker implementation has on real-time systems, however, there exists other factors that are of interest to investigate. There exists a need for understanding the impact of using Docker on multiple hosts, as CPSs typically contain multiple computing nodes. The computational overhead of using multiple hosts is identified in \cite{p1} in the context of cloud computing. However, a similar approach is needed within the context of CPSs, using experimental units that utilise a full implementation of a CPS such as the one implemented in this research.\\

To conclude, it is shown that the software requirements for the specific use case play the primary role for deciding whether Docker is safe or unsafe to implement into the software deployment architecture. This research has found that for vehicular CPSs which have scheduling precision, camera input and disk output performance as critical software requirements, Docker is a favourable choice for the deployment strategy. Docker is favourable due to the added value it brings to a software development project in the form of saved time, added safety and functionality, and portability among other things. However, for vehicular CPS projects which introduce additional critical requirements such as network performance, further research is required to understand how Docker behaves when network performance between separate computer nodes is part of the critical requirements. This is particularly interesting to explore as there may exist additional scheduling precision overhead of the CPS when introducing significant networking interrupts onto a system executing the CPS. Therefore, it is required to understand the behaviour during such a scenario before drawing conclusions to whether Docker performs appropriately as a deployment strategy in a networked environment.