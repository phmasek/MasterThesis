\iffalse \bibliography{include/backmatter/magnus,include/backmatter/philip} \fi
\chapter{Conclusion \& Future Work} \label{section:conclusion}

This research seeks to understand the impact Docker brings when utilised in the deployment strategy for systems which enable self-driving capabilities in vehicles. The research was conducted in collaboration with Chalmers Revere who are develing a self-driving truck. The results from this research acts as support in the decision making process of using Docker containers for software deployment for further software development on the self-driving truck project. Previous literature \cite{p6,c2,p3,p4,p7,c1} shows promising results for containers as a deployment platform where the results present negligible overhead introduced by utilising Docker in the execution environment of the applications explored. However, previous research has not been able to build a case for the decision making specific for the domain of self-driving vehicles where the domain stresses the real-time requirements of vehicular cyber-physical systems. While the research conducted in \cite{p1} have explored how Docker behaves using cyclic test in a cloud computing environment with real-time requirements, it has not explored how Docker behaves in the context of a complete cyber-physical system. This is the gap this research has explored.\\

The findings of this research are in-line with previous research and show that Docker introduces negligible overhead to the cyber-physical system in both the controlled environment as well as the environment of the self-driving truck at Chalmers Revere lab. Furthermore, the results convey that choosing a correct \textit{kernel} for the system carry heavier importance in comparison to choosing Docker or native execution for the deployment strategy. Requiring a real-time kernel is typical for rea-time systems. The results show a considerably more deterministic environment when utilising an RT\_preempt kernel in comparison with to a vanilla kernel. This argues that Docker is not the predominant factor impacting the scheduling or IO performance for when designing the correct execution environment for a self-driving vehicle.\\

This research has intentionally narrowed the focus onto the specific context of implementing Docker as a deployment strategy for self-driving vehicles. The design of this research has aimed to find a generalizable understanding of how Docker impacts scheduling precision and camera/disk IO for the context used. Analysis has been carried out to understand how these performance factors behave while switching between different treatments, such as system load and kernel version. The performance factors scheduling precision and camera/disk IO build a foundation of understanding the impact of using Docker for real-time systems, however, there exists other factors that are of interest to investigate. There exists a need for understanding the impact of using Docker on multiple hosts, as CPSs typically contain multiple computing nodes. The computational overhead of using multiple hosts is identified in \cite{p1} in the context of cloud computing. However, a similar approach is needed from the context of CPS, using experimental units that relate to actual implementation components of a complete CPS.\\

To conclude, it is shown that the software requirements for the specific use case plays the primary role for deciding whether Docker is safe or unsafe to implement into the software deployment architecture. This research has found that for vehicular CPSs having the scheduling precision, camera input and disk output performance as critical software requirements, Docker is a favourable choice for the deployment strategy. Docker is favourable due to the added value it brings to a software development project in the form of saved time, added safety and functionality, and portability among other things. However, for vehicular CPS projects which introduce additional critical requirements such as network performance, further research is required to understand how Docker behaves when network performance between separate computer nodes is part of the critical requirements. There may also exist additional scheduling precision overhead of the CPS when introducing significant networking interrupts onto a system executing the CPS. Therefore, it is required to understand the behaviour during such a scenario before drawing conclusions to whether Docker performs appropriately as a deployment strategy in a networked environment.



% Decision makers inquiring whether Docker is safe to implement in a cyber-physical system within the domain of self-driving vehicles, need firstly understand how Docker benefits the software engineering procedures for the project and the requirements for the intended system.



% During the research project, the team experienced that porting existing software into a Dockerized deployment strategy was shown to carry low transferability cost, where the porting of the full system took roughly 30 hours of work.











%performance while running several applications on separate machines in a network cluster. Current work \cite{conf/cloudcom/MaoHPWCCH15} indicates that network performance is worsened while running real-time applications in Docker containers. Future work is therefore required to understand how Docker impacts the network performance for real-time systems in the context of autonomous self-driving vehicles.\\





%Future work need to investigate alternatives such as running the application in a Jailhouse environment to understand the impact Docker has on the determinism of a real-time system in comparison with an environment specifically designed for real-time systems. This, to see whether or not Docker is comparable with an environment design solely for real-time system implementation.\\

%It is interesting to see the behaviour of Docker in a hardware environment utilising an ARM CPU. Furthermore, as Docker is continuously transformed with updates and patches, new configuration options arise. Future work is needed to further understand which Docker configuration is optimal for real-time systems.\\



% \begin{enumerate}
% \item jailhouse?
% \item running mutiple containers on different hosts as it has been seens that there is significant overhead regarding this (maybe something about the truck having multiple computing nodes) %http://ieeexplore.ieee.org/stamp/stamp.jsp?tp=&arnumber=7396222
% \item running with the new LTS OS that has zfs storage drivers
% \item windows and mac native support 
% \item doing the experiment on arm 



% \end{enumerate}